\documentclass{article}
\usepackage{ctex}
\title{优化理论 作业2}
\author{} 
\date{}
\usepackage[a4paper,left=10mm,right=10mm,top=15mm,bottom=15mm]{geometry}  
\begin{document}
\section*{课堂习题}
\noindent
\textbf{题目:}求$\mathop{min}\limits_{\vec{x}}\vert\vert\vec{x}\vert\vert_2\ (\textbf{s.t.}A\vec{x}=b)$,其中线性方程组$A\vec{x}=b$有无穷多组解.\\
\textbf{解:}不妨设$A\in\mathcal{R}^{m\times n},x\in\mathcal{R}^{n\times 1},b\in\mathcal{R}^{m\times 1}$,\\
则由题意得$m<n,r(A)=m$.\\
原题目可归结为求$\mathop{min}\limits_{\vec{x}}\frac{1}{2}\vert\vert\vec{x}\vert\vert_2^2 =\frac{1}{2}x^Tx$.\\
引入拉格朗日函数$L(x,\lambda)=\frac{1}{2}x^Tx-\lambda(Ax-b)$,其中参数$\lambda\in\mathcal{R}^{m\times 1}$.\\
令$\frac{\partial L}{\partial x}=x - A^T \lambda=0$,可得
\begin{equation}
    x=A^T\lambda
\end{equation}
令$\frac{\partial L}{\partial \lambda}=-(Ax-b)=0$,可得
\begin{equation}
    Ax=b
\end{equation}
下面联合方程(1)(2)解出$L$的极小值点:\\
由(1)得,$Ax=AA^T\lambda$,则有\begin{equation}
    \lambda = (AA^T)^{-1}Ax
\end{equation}
代入(2)得,\begin{equation}
    \lambda = (AA^T)^{-1}b
\end{equation}
由(1)(4)得,$x=A^T(AA^T)^{-1}b$.\\
由于目标函数$L$只有唯一极小值点,$x=A^T(AA^T)^{-1}b$即为满足题意的最小值.
\end{document}
