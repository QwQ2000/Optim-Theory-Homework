\documentclass{article}
\usepackage{ctex}
\title{优化理论 作业5}
\author{} 
\date{}
\usepackage[a4paper,left=10mm,right=10mm,top=15mm,bottom=15mm]{geometry}  
\begin{document}
\section*{6.1}
\noindent
\textbf{(1)}令$d=\left(
    \begin{array}{c}
        1 \\
        -2 \\
    \end{array}
\right)$,则有$d^Tf(x^*)=-1<0$,违背了一阶必要条件,因此$x^*$必不为局部最优解.\\
\textbf{(2)}考察所有可能的$d=\left(
    \begin{array}{c}
        a \\
        b \\
    \end{array}
\right)$,由题意得$a\geq 0$,则必有一阶必要条件$d^Tf(x^*)\geq 0 $恒成立.\\
因此,$x^*$可能为局部最优解.\\
\textbf{(3)}显然,Hessian矩阵为正定矩阵,这满足二阶充分条件,因此$x^*$必为局部最优解.\\
\textbf{(4)}由\textbf{(2)}知,$x^*$满足一阶必要条件.令$d=\left(
    \begin{array}{c}
        0 \\
        1 \\
    \end{array}
\right)$,则有$d^TF(x^*)d=-1<0$,这违背了二阶必要条件,故$x^*$必不为局部最优解.
\end{document}
