\documentclass{article}
\usepackage{ctex}
\title{优化理论 作业1}
\author{} 
\date{}
\usepackage[a4paper,left=10mm,right=10mm,top=15mm,bottom=15mm]{geometry}  
\begin{document}
\section*{2.1}
\noindent 
由题意得$A\in \mathcal{R}^{m\times n}$,因此$A$可表示为
$\left(
    \begin{array}{c}
        \alpha_1 \\
        \alpha_2 \\
        \dots \\
        \alpha_m
    \end{array}
\right)$,其中$\alpha_1,\alpha_2,\dots,\alpha_m \in \mathcal{R}^n.$\\
由$r(A)=m$知$\alpha_1,\alpha_2,\dots,\alpha_m$线性无关.\\
假设$m>n$,即$m\geq n+1$,则由$n+1$个$n$维向量必线性相关可知$\alpha_1,\alpha_2,\dots,\alpha_m$线性相关.\\
这与$\alpha_1,\alpha_2,\dots,\alpha_m$线性无关相矛盾.
因此$m\leq n$
\section*{2.6}
\noindent 
令$A=\left(
    \begin{array}{cccc}
        1 & 1 & 2 & 1 \\
        1 & -2 & 0 & -1
    \end{array}
\right)$,由于$A\rightarrow \left(
    \begin{array}{cccc}
        1 & 1 & 2 & 1 \\
        0 & -3 & -2 & -2
    \end{array}
\right)$,故$r(A)=2$.\\
$\widetilde{A}=[A;b]=\left(
    \begin{array}{cccc|c}
        1 & 1 & 2 & 1 & 1\\
        1 & -2 & 0 & 1 & -2
    \end{array}
\right)$,由于$\widetilde{A} \rightarrow \left(
    \begin{array}{ccccc}
        1 & 1 & 2 & 1 & 1  \\
        0 & -3 & -2 & -2 & -3
    \end{array}
\right)$,故$r(A,b)=2$\\
因此,这一线性方程组有解.\\
令$x_3=d_3,x_4=d_4$,$B=\left(
    \begin{array}{cc}
        1 & 1 \\
        1 & -2 
    \end{array}
\right)$,则有$\left(\begin{array}{c}x_1\\x_2\end{array}\right)=
B^{-1}[\left(\begin{array}{c}1\\-2\end{array}\right)-d_3
\left(\begin{array}{c}2\\0\end{array}\right)-d_4
\left(\begin{array}{c}1\\-1\end{array}\right)]=\\
\left(
    \begin{array}{cc}
        \frac{2}{3} & \frac{1}{3} \\
        \frac{1}{3} & -\frac{1}{3} 
    \end{array}
\right)\left(\begin{array}{c}1-2d_3-d_4\\d_4-2\end{array}\right)=
\left(\begin{array}{c}-\frac{4}{3}d_3-\frac{1}{3}d_4\\1-\frac{2}{3}d_3-\frac{2}{3}d_4\end{array}\right)$\\
因此,方程组的通解为$\left(\begin{array}{c}x_1\\x_2\\x_3\\x_4\end{array}\right)=
\left(\begin{array}{c}-\frac{4}{3}d_3-\frac{1}{3}d_4\\1-\frac{2}{3}d_3-\frac{2}{3}d_4\\d_3\\d_4\end{array}\right)$
\maketitle
\end{document}
