\documentclass{article}
\usepackage{ctex}
\title{优化理论 作业3}
\author{} 
\date{}
\usepackage[a4paper,left=10mm,right=10mm,top=15mm,bottom=15mm]{geometry}  
\begin{document}
\section*{3.5}
\noindent
设新的一组基为$V=\{v_1,v_2,v_3,v_4\}$,对于任意非零向量$x\in\mathcal{R}^4$,线性变换的形式为$Ax=y$\\
在新的一组基下,$x'=Vx,y'=Vy$.\\
设新一组基下的线性变换矩阵为对角阵$\Lambda$,则有$\Lambda x'=y'$,即$V^{-1}\Lambda Vx=y=Ax$\\
显然,对矩阵$A$进行相似对角化,则相似变换矩阵即为所需的基变换矩阵.\\
特征方程$Ax=\lambda x$,则有$(\lambda E-A)x=0$,特征方程$|\lambda E-A|=0$解得特征值$\lambda_1=2,\lambda_2=3,\lambda_3=1,\lambda_4=-1$.\\
将特征值分别带入原特征方程,分别解得特征向量为$x_1 = \left(
    \begin{array}{c}
        0 \\
        0 \\
        1 \\
        0
    \end{array}
\right),x_2 = \left(
    \begin{array}{c}
        0 \\
        0 \\
        1 \\
        1
    \end{array}
\right),x_3=\left(
    \begin{array}{c}
        0 \\
        2 \\
        -9 \\
        1
    \end{array}
\right),x_4=\left(
    \begin{array}{c}
        24 \\
        -12 \\
        1 \\
        9
    \end{array}
\right)$.\\则所求的基即为$V=\{x_1,x_2,x_3,x_4\}$
\section*{4.1}
\noindent
必要性:由题意可知,$S$可表示为$\{x\in\mathcal{R}^n:Ax=b\},A\in\mathcal{R}^{m\times n },b\in\mathcal{R}^{m}$.\\
对于任意$x,y\in S,\alpha \in\mathcal{R}$,显然$Ax=b,Ay=b$.\\
因为$A[\alpha x+(1-\alpha)y]=Ay+\alpha Ax-\alpha Ay=b+\alpha b-\alpha b = b$,所以$\alpha x+(1-\alpha)y\in S$,必要性得证.\\
充分性:若$S$为空集,则原命题充分性得证.\\
当$S$不为空时,设$x_0\in S$,令集合$S_0 = S-x_0=\{x-x_0:x\in S\}$.\\
显然,$0\in S_0$,取任意$x,y\in S_0,\alpha\in\mathcal{R}$,则有\\
$\alpha  = \alpha x + (1-\alpha)0\in S_0$;\\
$\frac{1}{2}x+(1-\frac{1}{2})y\in S_0$,即$x+y\in S_0$.\\
因此$S_0$为子空间,则必有$S_0=\mathcal{N}(A)=\{x:Ax=0\}$.令$b=Ax_0$,则\\
$S=S_0+x_0=\{y+x_0:y\in\mathcal{N}(A)\}=\{y+x_0:Ay=0\}=\{y+x_0:A(y+x_0)=b\}=\{x:Ax+b\}$.\\
综上所述,原命题的充分性得证.
\end{document}
