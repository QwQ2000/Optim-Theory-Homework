\documentclass{article}
\usepackage{ctex}
\usepackage{amsmath}
\usepackage{bm}

\boldmath
\title{优化理论 复习}
\author{QwQ2000} 
\date{2021年6月21日}
\usepackage[a4paper,left=10mm,right=10mm,top=15mm,bottom=15mm]{geometry}  
\begin{document}
\maketitle
\noindent
\section{Cauchy-Schwarz不等式及其推广}
Cauchy-Schwarz不等式: 
\begin{equation*}
    <x,y>\leq||x|| ||y||
\end{equation*}

其中二范数为平方和开根.\\

证明三角不等式
\begin{equation*}
    ||x+y||\leq||x||+||y||
\end{equation*}

\textbf{证明:}
\begin{align*}
    ||x+y||^2&=||x||^2+2<x,y>+||y||^2\\
             &\leq ||x||^2+2||x|| ||y||+||y||^2\\
             &=(||x||+||y||)^2    
\end{align*}

当$<x,y>=0$时,可得勾股定理:
\begin{equation*}
    ||x+y||^2\leq ||x||^2+||y||^2
\end{equation*}

\section{最小二乘典题}
\textbf{题目:}求$\mathop{min}\limits_{\vec{x}}\vert\vert\vec{x}\vert\vert_2\ (\textbf{s.t.}A\vec{x}=b)$.

\textbf{解:}原题目可归结为求$\mathop{min}\limits_{\vec{x}}\frac{1}{2}\vert\vert\vec{x}\vert\vert_2^2 =\frac{1}{2}x^Tx$.\\

引入拉格朗日函数$L(x,\lambda)=\frac{1}{2}x^Tx-\lambda(Ax-b)$,其中参数$\lambda\in\mathcal{R}^{m\times 1}$.\\

令$\frac{\partial L}{\partial x}=x - A^T \lambda=0$,可得
\begin{equation*}
    x=A^T\lambda
\end{equation*}

令$\frac{\partial L}{\partial \lambda}=-(Ax-b)=0$,可得
\begin{equation*}
    Ax=b
\end{equation*}

下面联合方程(1)(2)解出$L$的极小值点:\\

由(1)得,$Ax=AA^T\lambda$,则有\begin{equation*}
    \lambda = (AA^T)^{-1}Ax
\end{equation*}

代入(2)得,\begin{equation*}
    \lambda = (AA^T)^{-1}b
\end{equation*}

由(1)(4)得,$x=A^T(AA^T)^{-1}b$.\\

由于目标函数$L$只有唯一极小值点,$x=A^T(AA^T)^{-1}b$即为满足题意的最小值.\\

\textbf{题目:}求$\mathop{min}\limits_{\vec{x}}||Ax-b||_2$.

\textbf{解:}原问题可归结为求$f(x)=||Ax-b||^2_2$的最小值.

令$\nabla f(x)=A^T(Ax-b)=0$,解得$x=(A^TA)^{-1}A^Tb$

几何性质:$Ax^*$是$b$在$A$的列向量张成空间上的投影.

\section{线性变换与特征值典题}
\textbf{题目:}对于线性变换$A$,找到一组基,使得这组基下$A$为对角阵.

\textbf{解:}对于任意自然基下的向量$x,y$,有$Ax=y$.

假设新基就是$A$的特征向量,有$V=\{v_1,v_2,...,v_n\}$,则新基下$A$的表示形式为$diag(\lambda_1,\lambda_2,...,\lambda_n)$.

$x$在自然基下的坐标为$\left(
    \begin{array}{c}
        x_1 \\
        \dots \\
        x_n
    \end{array}
\right)$,则$x$在$\{v_1,v_2,...,v_n\}$下坐标为$\left(
    \begin{array}{c}
        a_1 \\
        \dots \\
        a_n
    \end{array}
\right)$.

$y$在自然基下的坐标为$\left(
    \begin{array}{c}
        y_1 \\
        \dots \\
        y_n
    \end{array}
\right)$,则$y$在$\{v_1,v_2,...,v_n\}$下坐标为$\left(
    \begin{array}{c}
        b_1 \\
        \dots \\
        b_n
    \end{array}
\right)$.

则有$\{e_1,\dots,e_n\}\left(
    \begin{array}{c}
        x_1 \\
        \dots \\
        x_n
    \end{array}
\right)=V\left(
    \begin{array}{c}
        a_1 \\
        \dots \\
        a_n
    \end{array}
\right)$

两端左乘$A$可得$A\left(
    \begin{array}{c}
        x_1 \\
        \dots \\
        x_n
    \end{array}
\right)=AV\left(
    \begin{array}{c}
        a_1 \\
        \dots \\
        a_n
    \end{array}
\right)$

即$A\left(
    \begin{array}{c}
        x_1 \\
        \dots \\
        x_n
    \end{array}
\right)=Vdiag(\lambda_1,\dots,\lambda_n)\left(
    \begin{array}{c}
        a_1 \\
        \dots \\
        a_n
    \end{array}
\right)$

对于$y$有$\{e_1,\dots,e_n\}\left(
    \begin{array}{c}
        y_1 \\
        \dots \\
        y_n
    \end{array}
\right)=V\left(
    \begin{array}{c}
        b_1 \\
        \dots \\
        b_n
    \end{array}
\right)$

因为$A\left(
    \begin{array}{c}
        x_1 \\
        \dots \\
        x_n
    \end{array}
\right)=\{e_1,\dots,e_n\}\left(
    \begin{array}{c}
        y_1 \\
        \dots \\
        y_n
    \end{array}
\right)$

所以$Vdiag(\lambda_1,\dots,\lambda_n)\left(
    \begin{array}{c}
        a_1 \\
        \dots \\
        a_n
    \end{array}
\right)=V\left(
    \begin{array}{c}
        b_1 \\
        \dots \\
        b_n
    \end{array}
\right)$,消去$V$可证在$V$下线性变换$A$对应的矩阵为对角阵.
\section{二次型典题}
\textbf{题目:}$x$经正交变换变换为$y$,求这二者长度关系?

\textbf{解:}由题意得$Q^TQ=I,y=Qx$.

则有$||y||^2 =<Qx,Qx>=x^TQ^TQx=x^TIx+x^Tx=||x||^2$

因此$x$与$y$等长.\\

\textbf{题目:}求$\mathop{min}\limits_{\vec{x}}x^TQx, x_1+x_2+x_3=1$.

\textbf{解:}由于$x^TQx$为二次型,故函数$f(x)=x^TQx$可导.

令$\nabla f(x)=2Qx=0$,由于$Q$为正定阵,则必有$Q^{-1}$存在,解得$x=0$.

但是$x=0$不满足$x_1+x_2+x_3=1$条件,则原问题转化为在平面$x_1+x_2+x_3$上找离$x=0$最近的点.
\section{矩阵范数的性质}
\begin{itemize}
    \item 非零矩阵范数>0,零矩阵范数为0
    \item $c\in \mathcal{R},||cA||=|c| ||A||$
    \item 三角不等式$||A+B||\leq||A||+||B||$
    \item $||AB||\leq||A|| ||B||$
\end{itemize}

\section{凸函数的定义}
对于$\forall x,y\in D,\forall\alpha\in [0,1]$
\begin{itemize}
    \item $\alpha f(x)+(1-\alpha)f(y)\geq f(\alpha x+(1-\alpha)y)$
    \item $f(y)\geq f(x)+\nabla f(x)(y-x)$
    \item $f''(x)\geq 0$或Hessian矩阵半正定
\end{itemize}

\section{局部最优解判定的条件}
\subsection{First Order Necessary Condition(FONC)}
对于需要判定的局部最优解$x^*$, $d$为定义域内任意可行的方向,则条件为
\begin{equation*}
    \nabla f(x^*)\cdot d \geq 0
\end{equation*}
\subsection{Second Order Necessary Condition(SONC)}
存在$d$使得$f(x^*)\cdot d=0$,则
\begin{equation*}
    d^T\nabla^2f(x^*)d \geq 0
\end{equation*}
\subsection{Second Order Sufficient Condition(SOSC)}
\begin{equation*}
    \nabla f(x^*)=0,\nabla^2f(x^*)\succ 0
\end{equation*}

即梯度为0,Hessian矩阵正定
\section{一维最优化方法——黄金分割法}
黄金分割法的Python代码:
\begin{verbatim}
    def gsec(f,a,b,eps = 1e-3):
        ratio = (np.sqrt(5) - 1) / 2
        while (b - a) > eps:
            a2,a1 = a + (b - a) * ratio,b - (b - a) * ratio #左侧和右侧的黄金分割点
            #保留值较小的一个分割点
            if f(a1) <= f(a2): 
                b = a2
            else:
                a = a1
        return (a + b) / 2
\end{verbatim}

\section{牛顿迭代法及其推导}
对于目标函数$f(x)$,在$x^k$处的近似可以用$f(x)$的二阶泰勒展开表示.

这一近似的具体形式为
\begin{equation*}
    g(x)=f(x^k)+\nabla f(x^k)(x-x^k)+\frac{1}{2}(x-x^k)^TH_f(x^k)(x-x^k)
\end{equation*}

求$g(x)$最小值,令$\nabla g(x)=0$解得$x=x^k-H^{-1}_f(x^k)\nabla f(x^k)$

则牛顿迭代法公式为$x^{k+1}=x^k-H^{-1}_f(x^k)\nabla f(x^k)$

\section{梯度下降法}
\textbf{证明负梯度方向是目标函数下降最快的方向:}


由柯西不等式得,目标函数在任意方向上$d(||d||=1)$的增长率
\begin{equation*}
    <\nabla f(x),d>\leq ||\nabla f(x)||\cdot ||d||=\nabla f(x)
\end{equation*}

当且仅当$d=\frac{\nabla f(x)}{||\nabla f(x)||}$时,增长率最大.
因此,负梯度方向目标函数下降最快.\\

\textbf{问题:}对于$i=1.5,y=0.5,w=0.8$,利用梯度下降法调整$w$的值使得$wi=y$.

梯度下降法的伪代码:
\begin{verbatim}
    i,w = 1.5,0.8
    y = 0.5
    lr = 0.001
    eps = 1e-6

    while 1:
        g = 2 * (i * w - y) * i #计算w的梯度,设损失函数为(i * w - y)**2
        if abs(g) < eps:
            break
        w = w - lr * g #进行梯度下降
\end{verbatim}

最速梯度下降法,利用一维最优化方法寻找最佳的步长,伪代码:
\begin{verbatim}
    x = x0
    while |grad(x)| > eps:
        a = argmin_{a>=0}(f(x - a * grad(x)))
        x = x - a * grad(x)
\end{verbatim}

\section{近端梯度}
近端梯度算子:
\begin{equation*}
    prox_{\lambda f}(v)=\mathop{argmin}\limits_{x}(f(x)+\frac{1}{2\lambda}||x-v||^2)
\end{equation*}

对于不可导函数$h(x)=f(x)+\lambda g(x)$,其中$f(x)$光滑,$prox_g(x)$易求得,
就可以用近端梯度方法优化不可导的目标函数.

有迭代公式:
\begin{equation*}
    x^{k+1}=prox_{\lambda^kg}(x^k-\lambda^k\nabla f(x^k))
\end{equation*}

\textbf{典型例题之LASSO问题:}最小化$f(x)=\frac{1}{2}||Ax-b||^2_2+\lambda||x||_1$\\

\textbf{解:}可导部分$g(x)=\frac{1}{2}||Ax-b||,\nabla g(x)=A^T(Ax-b)$.

则近端梯度法的迭代公式为
\begin{equation*}
    x^{k+1}=prox_{\lambda||x||_1}(x^k-\lambda\nabla g(x^k))    
\end{equation*}

对于1-范数$||x||_1$,近端梯度算子可用软阈值函数表示,即
\begin{equation*}
    x^{k+1}=\mathcal{S}_{\lambda}(x^k-\lambda\nabla g(x^k))
\end{equation*}

其中软阈值函数
\begin{equation*}
    \mathcal{S}_\lambda(x)=\left\{
    \begin{aligned}
    &x-\lambda,&x\geq\lambda\\
    &0,&-\lambda<x<\lambda\\
    &x+\lambda,&x\leq -\lambda
    \end{aligned}
    \right.
\end{equation*}
\end{document}